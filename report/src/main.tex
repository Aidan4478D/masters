\documentclass[12pt]{article}
\usepackage[margin=1in]{geometry}
% \usepackage{times}
% \usepackage{newtxtext,newtxmath}
\usepackage{lmodern}
\usepackage{setspace}
\usepackage[backend=biber,style=ieee]{biblatex}
% \usepackage[backend=bibtex,style=ieee]{biblatex}
\usepackage{indentfirst}
\setlength{\parindent}{0pt} % 15pt to put indent
\setlength{\parskip}{0.75em}

\onehalfspacing

\addbibresource{references.bib}

\begin{document}
\pagenumbering{gobble}

% Center everything on the page
\begin{titlepage}
    \centering
    {THE COOPER UNION FOR THE ADVANCEMENT OF SCIENCE AND ART}\\
    {ALBERT NERKEN SCHOOL OF ENGINEERING}
    \vspace*{2in}
    
    {\Huge \textbf{Computer Vision for Vehicle Emission Estimation}}\\[0.75em]
    {\normalsize By}\\[0.75em]
    {\large Aidan Cusa}\\[15em]
    
    {A thesis submitted in partial fulfillment of the requirements for the degree of}\\
    {Master of Engineering}\\[2em]

    {Advisor}\\[1em]
    {Carl Sable}
\end{titlepage}
\clearpage
% \setcounter{page}{0}

\newpage
\centering
\vspace*{1in}
{THE COOPER UNION FOR THE ADVANCEMENT OF SCIENCE AND ART}\\[2em]
{ALBERT NERKEN SCHOOL OF ENGINEERING}\\[10em]

\raggedright
{This thesis was prepared under the direction of Carl Sable and has received 
approval. It was submitted to the Dean of the School of Engineering and the 
full Faculty, and was approved as partial fulfillment of the requirements 
for the degree of Master of Engineering.}\\[6em]


\raggedleft
\begin{tabular}{@{}p{.5in}p{4in}@{}}
& \hrulefill \\
& Barry L. Shoop, Ph.D, P.E. - Date \\
& Dean, Albert Nerken School of Engineering\\
\end{tabular}\\[8em]

\raggedright
\begin{tabular}{@{}p{0in}p{4in}@{}}
& \hrulefill \\
& Carl Sable, Ph.D - Date \\
& Candidate's Thesis Advisor\\
\end{tabular}

\newpage
\section*{Acknowledgement}
{I would like to express my sincere gratitude to my mentor Professor Carl Sable
for his guidance, patience, and encouragement throughout this project. I would 
also like to thank my teammates, Sun Kongsonthana, and Minahil Bakhtawar for 
their collaboration, creativity, and teamwork during the development of the 
Remote Micro-Robot middleware.}\\[1em]

{I am grateful to the College of Design and Engineering at the National University 
of Singapore for providing the resources and support that made this work possible. 
Finally, I would like to thank The Cooper Union for providing me with this 
incredible opportunity to represent the school abroad.}

\pagenumbering{roman}
\setcounter{page}{1}

\newpage
\begin{abstract}
    Hello
\end{abstract}

\newpage
\tableofcontents

% \newpage
% \section*{List of Figures}
%
% \newpage
% \section*{Table of Nomenclature}

\newpage
\pagenumbering{arabic}
\setcounter{page}{1}

\section{Introduction}

\subsection{Motivation}

Urban air pollution remains a major public-health and environmental challenge. 
The World Health Organization (WHO) estimates that around 99\% of the world's 
population breathes air containing pollutant concentrations that exceed WHO 
guideline limits, and that air pollution is responsible for more than 7.9 
million premature deaths annually. [cite arabian peninsula and WHO. New State 
of Global Air 2025 Report] Additionally, it is linked to a broad range of 
chronic and acute outcomes, including cardiopulmonary disease and asthma, and 
it also contributes to environmental harms such as global warming and weather 
variability [cite arabian peninsula, New State of Global Air 2025 Report, 
Transport impacts on atmosphere and climate]. Road vehicles are central to both 
the climate and health aspects of the problem. The Intergovernmental Panel on 
Climate Change (IPCC) Sixth Assessment Report Working Group III notes that, in 
2019, direct greenhouse gas (GHG) emissions from the transport sector accounted
for 23\% of global energy-related CO2 emissions. Of these, 70\% came from road 
vehicles [cite IPCC]. In urban areas, transportation emissions are particularly
consequential as they are concentrated along the very corridors where people 
live, work, and travel.[cite arabian peninsula]

New York City (NYC) is an intersection between dense population, heavy traffic
activity, and measurable health burdens. As of July 2024, there are 8.48 million 
residents in all five boroughs, with (x amount of people near major roadway/
highway that is high risk exposure. ASK SABLE: should I try to find the number 
for this if I have the time? Like should I try to contribute as much as I can? 
I can’t find a number past 2007) While the typical passenger vehicle emits ~411
grams of CO2 per mile, it also emits pollutants that directly affect human 
health, especially fine particulate matter like PM2.5. NYC’s Department of 
Health and Mental Hygiene describes PM2.5 as “the most harmful air pollution 
for humans to breathe” because it can enter the bloodstream through the lungs, 
and it attributes substantial ongoing health impacts to current exposures [cite
NYC DOH]. Specifically, NYC reports that current overall PM2.5 levels from all 
sources contribute to approximately 2,000 deaths and 5,150 emergency department
visits and hospitalizations for respiratory and cardiovascular disease each 
year [cite NYC DOH]. Importantly for transportation policy and targeted 
mitigation, NYC further estimates that 14\% of PM2.5 comes from everyday car, 
bus, and truck traffic, and that PM2.5 pollution from traffic in the NYC region
contributes to an estimated 320 premature deaths and 870 emergency department 
visits, and hospitalizations each year in New York City [cite NYC DOH].  UCS's 
New York State also reports that, in New York State, the combined health and 
climate costs in 2015 attributable to passenger vehicles were about \$7.9 
billion, with health costs comprising roughly two-thirds of the total. 
[cite UCS]

\section{Background Research}

\section{Dataset}

\section{Methodology}

\section{Experimentation and Results}

\section{Discussion}

\section{Limitations}

\section{Future Work}

\section{Bibliography}

% \printbibliography

\end{document}

\section{Related Work}
\subsection{Traditional Air Quality Monitoring Methods}

This section reviews the evolution of methodologies for quantifying urban air
pollution, ranging from traditional stationary infrastructure to deep learning 
approaches. It evaluates the strengths and limitations of direct measurement 
approaches. It identifies the critical data gaps in spatial resolution and source 
attribution that necessitate the development of high-resolution, vision-based 
bottom-up inventories.

Rewrite:
Current approaches to urban air quality assessment are generally categorized into top-down and bottom-up methodologies. Top-down approaches infer emission fluxes from observed atmospheric concentrations, using data from satellite remote sensing or stationary monitoring networks to validate regional models. Conversely, bottom-up approaches calculate emissions at the source by combining activity data (e.g., traffic flow) with emission factors. This section reviews the evolution of these monitoring paradigms, evaluating the limitations of current top-down observation in resolving street-level heterogeneity and identifying the critical need for high-resolution, vision-based activity data to improve bottom-up inventories.

\subsubsection{Stationary Air Quality Monitoring Stations (AQMS)}
Regulatory-grade stationary networks are the standard for air quality 
assessment, providing high-quality, standardized data ideal for analyzing 
spatiotemporal trends across large regions [1]. However, the number of these 
fixed-site monitors are limited. For example, in U.S. urban areas with 
continuous monitoring, there are typically only 2-5 monitors per million 
people [2]. Consequently, they fail to capture hyperlocal, street-by-street 
variations in pollution, which is the scale at which human exposure and 
exceedances of ambient standards often occur [1], [2]. Research indicates 
that concentrations of pollutants such as ultrafine particles (UFP), black 
carbon (BC), and nitrogen oxides ($NO_x$) can vary by 5–8 times within 
individual city blocks and fluctuate significantly over distances of less than 
300 meters [1], [2]. While dispersion models and satellite remote sensing are 
often used to supplement stationary data, they face intrinsic resolution 
limitations (discussed further in Section 2.2). Thus, they cannot fully resolve
the fine-scale gradients (10–300 m) driven by local traffic emissions [2].

\subsubsection{Mobile Monitoring Platforms}
To address the spatial limitations of fixed networks, mobile monitoring 
(mounting reference-grade instruments on fleet vehicles) has emerged as a 
method to capture high-resolution spatial data [1]. Notable implementations 
include equipping Google Street View cars to repeatedly sample urban roadways, 
successfully revealing persistent pollution patterns and stable "hotspots" 
attributable to local sources [2]. This approach allows for the identification 
of source impacts without a priori assumptions about emission rates [1].

However, while mobile monitoring maximizes spatial coverage, it lacks temporal 
continuity [1]. Because a vehicle cannot measure all locations simultaneously, 
data must be aggregated over long periods or modeled to estimate averages [1]. 
Furthermore, measurement locations are inherently biased toward the roadway 
(on-road), which can differ from off-road concentrations (e.g., at home 
addresses) by 20–30\%, necessitating careful adjustment when estimating 
population exposure [1].

\subsubsection{Low-Cost Sensors and On-Board Diagnostics (OBD)}
A shifting paradigm in monitoring involves the deployment of portable, 
lower-cost sensors capable of reporting near real-time data [3]. These devices 
are generally categorized into gas-phase sensors (electrochemical or metal 
oxide) and particulate matter sensors (optical scattering) [3]. While these 
allow for denser deployment and community-based science, they suffer from 
significant data quality issues. As of April 2026, there are currently no 
commercially available direct-reading sensors for PM mass (relying instead
on light-scattering surrogates) or specific hazardous air pollutants, and 
performance criteria for non-regulatory use remain undefined [3].

Similarly, the mandatory equipping of heavy-duty vehicles with On-Board 
Diagnostic (OBD) systems offers a potential stream of "big data" for real-time 
emission monitoring [4]. While this provides an opportunity to supervise in-use
vehicles remotely, the data quality is often poor and lacks a comprehensive 
assessment, requiring substantial computational infrastructure to process 
effectively [4].

\subsection{Satellite-Based Remote Sensing}
While sensor networks provide point-based measurements, satellite remote 
sensing offers a top-down approach to air quality monitoring, playing a 
critical role in developing emission policies and forecasting regional 
air quality [5], [6].

\subsubsection{Resolution and Instrumentation}
The landscape of satellite monitoring has shifted with the deployment of
instruments like the Tropospheric Monitoring Instrument (TROPOMI) on the 
Sentinel-5P mission. TROPOMI provides daily global measurements at an 
unprecedented spatial resolution of approximately 3.5 $\times$ 5.5 km$^2$, 
allowing for the detection of trace gases ($NO_2$, $O_3$, $CO$) and their 
linkages to human activity [7], [8]. Complementary infrared sounders, such as 
IASI and AIRS, provide vertical profile data, though at coarser footprint 
resolutions (~12 km) [9]. Despite these advancements, even "high-resolution" 
satellite data remains too coarse (>3 km) to resolve the fine-scale gradients
(10–300 m) driven by urban traffic emissions, effectively smoothing out the 
street-level heterogeneity that defines urban exposure [2], [10].

\subsubsection{Physical and Environmental Constraints}
Beyond spatial resolution, satellite retrieval is limited by physical 
constraints, particularly regarding diurnal variation. "Top-down" observation 
is highly sensitive to the planetary boundary layer height; in the morning, 
a shallow boundary layer can mask surface-level pollutants, reducing 
sensitivityexactly when traffic emissions are often highest [10]. Furthermore, 
retrieval algorithms struggle with environmental variables such as surface 
reflectivity and viewing geometry (zenith angles), which can introduce 
artifacts into the data, particularly at high latitudes or in complex urban 
terrains [10].

\subsubsection{The Dependency on Bottom-Up Inventories}
To bridge the gap between regional satellite data and street-level reality, 
researchers rely on chemical transport models (CTMs) and data assimilation 
techniques. However, the efficacy of these models is fundamentally limited 
by the quality of their inputs. Recent reviews indicate that high-resolution
CTMs require detailed emission inventories (better than 1 km resolution) to
resolve small-scale dynamical features like urban heat islands and street 
canyon effects [10]. The current lack of such granular, dynamic "bottom-up"
inventories remains a bottleneck for effective satellite data assimilation,
validating the need for improved terrestrial emission modeling [10], [11].


\section{Introduction}

\subsection{Motivation}

Urban air pollution remains one of the most critical public health and 
environmental challenges of the twenty-first century. The World Health 
Organization (WHO) estimates that approximately 99\% of the global population 
breathes air containing pollutant concentrations that exceed guideline limits 
\cite{world_health_organization_air_nodate}, \cite{health_effects_institute_state_2025}. In 2023 alone, air pollution contributed to 7.9 million deaths
worldwide, with 86\% of these attributed to noncommunicable diseases such as 
cardiopulmonary conditions and lung cancer \cite{health_effects_institute_state_2025}. Beyond its immediate health 
impacts, atmospheric pollution catalyzes broader environmental degradation. 
Short-lived climate pollutants and greenhouse gases contribute to global 
warming, weather variability, and the intensification of the water cycle, 
which in turn threatens infrastructure and coastal settlements \cite{uherek_transport_2010}, \cite{irankunda_systematic_2025}.

Road vehicles are the central driver of both the climate and health aspects of 
this crisis. The Intergovernmental Panel on Climate Change (IPCC) Sixth 
Assessment Report notes that direct greenhouse gas (GHG) emissions from the 
transport sector accounted for 23\% of global energy-related $\mathrm{CO_2}$ emissions in 
2019 \cite{ipcc_climate_2022}. Of these, 70\% originated specifically from road vehicles [5]. 
Without intervention, transport emissions are projected to double by 2050, 
driven by increasing vehicle ownership in developing economies \cite{creutzig_transport_2015}. In urban 
environments, these emissions are particularly consequential because they are 
released at ground level along the very corridors where people live, work, and 
commute, creating immediate exposure risks \cite{uherek_transport_2010}.

New York City (NYC) serves as a critical intersection of dense population, 
heavy traffic activity, and measurable health burdens. As of July 2024, there 
are 8.48 million residents in all five boroughs, with (x amount of people near
major roadway/highway that is high risk exposure. ASK SABLE: should I try to 
find the number for this if I have the time? Like should I try to contribute 
as much as I can? I can’t find a number past 2007) \cite{nyc_department_of_city_planning_population_2024}. While the typical passenger
vehicle emits approximately 411 grams of $\mathrm{CO_2}$ per mile \cite{us_epa_greenhouse_2016}, it also releases 
pollutants toxic to human health, most notably fine particulate matter ($\mathrm{PM_{2.5}}$).
This pollutant is described by the NYC Department of Health and Mental Hygiene 
as "the most harmful air pollution for humans to breathe" as $\mathrm{PM_{2.5}}$ penetrates 
deep into the lungs and enters the bloodstream, circulates throughout the body,
and causes systemic inflammation \cite{nyc_department_of_health_and_mental_hygiene_traffic_nodate}.

The local impact of these emissions is quantifiable and severe. In New York 
City, $\mathrm{PM_{2.5}}$ levels from all sources contribute to approximately 2,000 deaths 
and 5,150 emergency department visits annually \cite{nyc_department_of_health_and_mental_hygiene_traffic_nodate}. 14\% of local $\mathrm{PM_{2.5}}$ is 
directly attributable to everyday car, bus, and truck traffic, resulting in 
an estimated 320 premature deaths and 870 hospitalizations each year \cite{nyc_department_of_health_and_mental_hygiene_traffic_nodate}. In 
New York State, the combined health and climate costs attributable to passenger
vehicles totaled approximately \$7.9 billion in 2015, with health-related costs 
comprising two-thirds of that total \cite{pinto_de_moura_inequitable_2019}.

Despite the well-documented impacts of transportation emissions, effective 
mitigation is hindered by the difficulty of accurate, timely measurement. 
Emissions from road vehicles are not distributed uniformly across space or 
time, but rather concentrated in specific "hotspots," vary by time of day, 
and depend heavily on vehicle composition. For instance, heavy-duty vehicles 
contribute disproportionately to nitrogen oxides (NOx) and particulate matter 
relative to light-duty cars \cite{wang_high-resolution_2025}, while congestion increases per-mile emissions
through idling and stop-and-go driving. Consequently, understanding not only 
total vehicle counts but also the specific composition of traffic flows is 
critical for estimating localized emissions burdens.

Traditional emissions estimators, however, often rely on aggregate fleet 
statistics and modeled travel demand applied at coarse spatial scales. While 
indispensable for long-term planning, these methods often lack the granularity
required to evaluate street-level exposure disparities. Similarly, fixed 
air-quality monitoring stations measure ambient concentrations but cannot 
directly attribute pollutants to specific vehicle classes or immediate traffic
patterns. Furthermore, monitoring infrastructure is unevenly distributed 
globally. High-income regions typically possess denser monitoring networks,
whereas many low- and middle-income countries lack adequate coverage, leaving
large populations without reliable local air quality data \cite{carvalho_air_2016}.

- can put better citations for here too

To address these limitations, this thesis proposes leveraging existing video
infrastructure as a scalable monitoring tool. New York City maintains an
extensive network of publicly accessible traffic cameras, offering a unique
opportunity to generate high-resolution data without deploying new hardware.
By applying computer vision techniques to identify vehicle classes and counts,
combined with established emission factors, it is possible to generate 
spatially resolved estimates of traffic-related emissions. This approach does
not replace traditional air-quality monitoring but complements it, providing a 
granular view of how traffic composition drives environmental health impacts. 
Moreover, this methodology offers a potential blueprint for regions with 
limited monitoring resources, suggesting that computer vision-based estimation 
could become an accessible, scalable tool for global environmental health 
monitoring.


